\setlength{\parindent}{10ex}

\section{Future Challenges for Parallel Computing and HPC}

\subsection{Scalable Energy Consumption}
Scalable Energy is a serious problem for the future of HPC.
Processors are slowly approaching a point where it is impossible to fit more transitors on a peice of silicon.
This limits the future efficency gains that processors can recieve. 
This will cause the energy scaling on a HPC cluster to be linear, and potentially unsustainable.
This is unless, we create new and plentiful energy supplies, or innovate energy efficency on processors.

\subsection{Big Data Processing}
Big Data processing is a prominent use case for HPC and Parallel Computing. 
However, Big data is a ever increasing problem.
As the problems increase, so do the demands for infrastructure.
Managing distributed memory, storage, and interconnection in the cluster becomes an increasingly difficult problem.

\subsection{Trends in Innovation}
Computing innovation in the 20th century was driven by the needs of the research and scientific communities.
This directly led to innovation in parallel architectures for these communities.
The 21st century is changing the paradigm on computing innovation.
Consumer electronics have been driving innovation.
This leads to the needs of that domain taking presidence.
There is a need for new algorithms to overcome new problems.
However, the new source of innovation threatens to neglect innovation in the HPC domain.
